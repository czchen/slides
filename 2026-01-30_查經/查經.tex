\documentclass{beamer}

\usetheme{CambridgeUS}
\usecolortheme{orchid}

\usepackage{polyglossia}
\setmainlanguage{chinese}
\setotherlanguage{hebrew}
\setotherlanguage{greek}
\setotherlanguage{english}

\usepackage{fontspec}
\usefonttheme{professionalfonts}

\newfontfamily\chinesefont{Noto Sans CJK TC}
\newfontfamily\cjkfontsf{Noto Sans CJK TC}
\newfontfamily\cjkfonttt{Noto Sans CJK TC}

\newfontfamily\hebrewfont[Script=Hebrew]{David CLM}
\newfontfamily\hebrewfontsf[Script=Hebrew]{David CLM}

\newfontfamily\englishfont{Noto Sans}

\usepackage[normalem]{ulem}
\usepackage{booktabs}

%%%%%%%%%%%%%%%%%%%%%%%%%%%%%%%%%%%%%%%%%%%%%%%%%%%%%%%%%%%%%%%%%%%%%%%%%%%%%%%%
% References
%%%%%%%%%%%%%%%%%%%%%%%%%%%%%%%%%%%%%%%%%%%%%%%%%%%%%%%%%%%%%%%%%%%%%%%%%%%%%%%%

\usepackage[backend=biber,style=numeric,sorting=none,sortcites=true]{biblatex}
\addbibresource{查經.bib}

%%%%%%%%%%%%%%%%%%%%%%%%%%%%%%%%%%%%%%%%%%%%%%%%%%%%%%%%%%%%%%%%%%%%%%%%%%%%%%%%
% Bible References
%%%%%%%%%%%%%%%%%%%%%%%%%%%%%%%%%%%%%%%%%%%%%%%%%%%%%%%%%%%%%%%%%%%%%%%%%%%%%%%%

\usepackage{bibleref-mouth}
\providebiblebookalias{Psalms}{Psalm} % Support Psalm as alias

\newcommand*{\setupchinesebible}[1]{
  % 五經
  \providebiblebook{#1}{Genesis}{創世記}
  \providebiblebook{#1}{Exodus}{出埃及記}
  \providebiblebook{#1}{Leviticus}{利未記}
  \providebiblebook{#1}{Numbers}{民數記}
  \providebiblebook{#1}{Deuteronomy}{申命記}

  % 前先知書
  \providebiblebook{#1}{Joshua}{約書亞記}
  \providebiblebook{#1}{Judges}{士師記}
  \providebiblebook{#1}{ISamuel}{撒母耳記上}
  \providebiblebook{#1}{IISamuel}{撒母耳記下}
  \providebiblebook{#1}{IKings}{列王紀上}
  \providebiblebook{#1}{IIKings}{列王紀下}

  % 後先知書 - 大先知書
  \providebiblebook{#1}{Isaiah}{以賽亞書}
  \providebiblebook{#1}{Jeremiah}{耶利米書}
  \providebiblebook{#1}{Ezekiel}{以西結書}

  % 後先知書 - 十二先知書
  \providebiblebook{#1}{Hosea}{何西阿書}
  \providebiblebook{#1}{Joel}{約珥書}
  \providebiblebook{#1}{Amos}{阿摩司書}
  \providebiblebook{#1}{Obadiah}{俄巴底亞書}
  \providebiblebook{#1}{Jonah}{約拿書}
  \providebiblebook{#1}{Micah}{彌迦書}
  \providebiblebook{#1}{Nahum}{那鴻書}
  \providebiblebook{#1}{Habakkuk}{哈巴谷書}
  \providebiblebook{#1}{Zephaniah}{西番雅書}
  \providebiblebook{#1}{Haggai}{哈該書}
  \providebiblebook{#1}{Zechariah}{撒迦利亞書}
  \providebiblebook{#1}{Malachi}{瑪拉基書}

  % 聖卷 - 詩歌書
  \providebiblebook{#1}{Psalms}{詩篇}
  \providebiblebook{#1}{Proverbs}{箴言}
  \providebiblebook{#1}{Job}{約伯記}

  % 聖卷 - 節期書
  \providebiblebook{#1}{SongofSongs}{雅歌}
  \providebiblebook{#1}{Ruth}{路得記}
  \providebiblebook{#1}{Lamentations}{耶利米哀歌}
  \providebiblebook{#1}{Ecclesiastes}{傳道書}
  \providebiblebook{#1}{Esther}{以斯帖記}

  % 聖卷
  \providebiblebook{#1}{Daniel}{但以理書}
  \providebiblebook{#1}{Ezra}{以斯拉記}
  \providebiblebook{#1}{Nehemiah}{尼希米記}
  \providebiblebook{#1}{IChronicles}{歷代志上}
  \providebiblebook{#1}{IIChronicles}{歷代志下}

  % 次經
  \providebiblebook{#1}{Tobit}{多俾亞傳}
  \providebiblebook{#1}{Judith}{友弟德傳}
  \providebiblebook{#1}{IMaccabees}{瑪加伯上}
  \providebiblebook{#1}{IIMaccabees}{瑪加伯下}
  \providebiblebook{#1}{Wisdom}{智慧篇}
  \providebiblebook{#1}{Ecclesiasticus}{德訓篇}
  \providebiblebook{#1}{Baruch}{巴錄書}

  % 福音書
  \providebiblebook{#1}{Matthew}{馬太福音}
  \providebiblebook{#1}{Mark}{馬可福音}
  \providebiblebook{#1}{Luke}{路加福音}
  \providebiblebook{#1}{John}{約翰福音}
  \providebiblebook{#1}{Acts}{使徒行傳}

  % 保羅書信
  \providebiblebook{#1}{Romans}{羅馬書}
  \providebiblebook{#1}{ICorinthians}{哥林多前書}
  \providebiblebook{#1}{IICorinthians}{哥林多後書}
  \providebiblebook{#1}{Galatians}{加拉太書}
  \providebiblebook{#1}{Ephesians}{以弗所書}
  \providebiblebook{#1}{Philippians}{腓立比書}
  \providebiblebook{#1}{Colossians}{歌羅西書}
  \providebiblebook{#1}{IThessalonians}{帖撒羅尼迦前書}
  \providebiblebook{#1}{IIThessalonians}{帖撒羅尼迦後書}
  \providebiblebook{#1}{ITimothy}{提摩太前書}
  \providebiblebook{#1}{IITimothy}{提摩太後書}
  \providebiblebook{#1}{Titus}{提多書}
  \providebiblebook{#1}{Philemon}{腓利門書}
  \providebiblebook{#1}{Hebrews}{希伯來書}

  % 其他書信
  \providebiblebook{#1}{James}{雅各書}
  \providebiblebook{#1}{IPeter}{彼得前書}
  \providebiblebook{#1}{IIPeter}{彼得後書}
  \providebiblebook{#1}{IJohn}{約翰一書}
  \providebiblebook{#1}{IIJohn}{約翰二書}
  \providebiblebook{#1}{IIIJohn}{約翰三書}
  \providebiblebook{#1}{Jude}{猶大書}

  % 啟示錄
  \providebiblebook{#1}{Revelation}{啟示錄}
}

% Fix \textendash
\makeatletter
\providebiblestyle{chinese-text}{\standardbiblestyle{chinese-text}
  {\ }{:}{; }{;}{,}{\textendash}
  {\brm@number@arabic}{\brm@number@arabic}
{#1}{#2}{#3}}
\makeatother

\setupchinesebible{chinese-text}

\providebiblegatewayurl{biblegateway-CUV}{CUV}
\providebiblegatewaystyle{chinese}{biblegateway-CUV}{chinese-text}
\setupchinesebible{chinese}

\setbiblestyle{chinese}

%%%%%%%%%%%%%%%%%%%%%%%%%%%%%%%%%%%%%%%%%%%%%%%%%%%%%%%%%%%%%%%%%%%%%%%%%%%%%%%%
% Content
%%%%%%%%%%%%%%%%%%%%%%%%%%%%%%%%%%%%%%%%%%%%%%%%%%%%%%%%%%%%%%%%%%%%%%%%%%%%%%%%

\title{信友團契 2026 查經 Lesson 1}
\subtitle{\bibleref{Nb 1:1-18}:神的秩序與同在}
\date{2026-01-30}

\begin{document}

\begin{frame}
  \titlepage
\end{frame}

\begin{frame}
  \frametitle{\bibleref{Nb 1:1-5}}
  \textbf{1}\ 接着,上主在\uline{西奈}曠野的會幕中對\uline{摩西}說話,這是在\uline{以色列人}出\uline{埃及}地之後第二年的二月一日。他說:
  \textbf{2}\ 「你要按宗族、按父家登記\uline{以色列}全會眾的人數,所有男丁都按名字的數目一一登記。
  \textbf{3}\ 在\uline{以色列}中凡是二十歲以上、能出去作戰的,你和\uline{亞倫}都要按着軍隊數點他們。
  \textbf{4}\ 每支派還要有一人與你們一起,那人須是他父家的首領。
\end{frame}

\begin{frame}
  \frametitle{\bibleref{Nb 1:5-16}}
  \textbf{5}\ 這些就是與你們站在一起之人的名字:屬\uline{呂便}的,是\uline{示丟珥}的兒子\uline{以利蓿};
  \textbf{6}\ 屬\uline{西緬}的,是\uline{蘇利沙代}的兒子\uline{示路蔑};
  \textbf{7}\ 屬\uline{猶大}的,是\uline{亞米拿達}的兒子\uline{拿順};
  \textbf{8}\ 屬\uline{以薩迦}的,是\uline{蘇押}的兒子\uline{拿坦業};
  \textbf{9}\ 屬\uline{西布倫}的,是\uline{希倫}的兒子\uline{以利押};
  \textbf{10}\ 屬\uline{約瑟}兒子\uline{以法蓮}的,是\uline{亞米忽}的兒子\uline{以利沙瑪};屬\uline{瑪拿西}的,是\uline{比大蓿}的兒子\uline{迦瑪列};
  \textbf{11}\ 屬\uline{便雅憫}的,是\uline{基多尼}的兒子\uline{亞比但};
  \textbf{12}\ 屬\uline{但}的,是\uline{亞米沙代}的兒子\uline{亞希以謝};
  \textbf{13}\ 屬\uline{亞設}的,是\uline{俄蘭}的兒子\uline{帕結};
  \textbf{14}\ 屬\uline{迦得}的,是\uline{丟珥}的兒子\uline{以利雅薩};
  \textbf{15}\ 屬\uline{拿弗他利}的,是\uline{以南}的兒子\uline{亞希拉}。
  \textbf{16}\ 這些從會眾裏選召出來的人,是他們父族眾支派的領袖,都是\uline{以色列}眾族的首領。」
\end{frame}

\begin{frame}
  \frametitle{\bibleref{Nb 1:17-19}}
  \textbf{17}\ \uline{摩西}和\uline{亞倫}帶着這些記名的人,
  \textbf{18}\ 在二月一日聚集了全會眾,按宗族、按父家,將二十歲以上的人,按名字的數目一一登記。
  \textbf{19}\ 照上主所吩咐\uline{摩西}的,他在\uline{西奈}曠野數點了他們。
\end{frame}

\begin{frame}
  \frametitle{前情提要 - \bibleref{Nb 24-40}\ 的交錯對稱\parencite{Chiasm_Exodus_24-40}}
  \begin{itemize}
    \item \textbf{A} 上主的榮耀停駐在\uline{西奈}山頂 (\bibleref{Ex 24:12-18})
    \item \textbf{B} 會幕的樣式和祭師的條例 (\bibleref{Ex 25-30})
    \item \textbf{C} 會幕的技工和材料 (\bibleref{Ex 31:1-11})
    \item \textbf{D} 安息日的規定 (\bibleref{Ex 31:12-17})
    \item \textbf{E} 背約,金牛犢事件 (\bibleref{Ex 31:18-33:11})
    \item \textbf{X} 上主的榮耀 (\bibleref{Ex 33:12-34:9})
    \item \textbf{E`} 重新立約 (\bibleref{Ex 34:10;35})
    \item \textbf{D`} 安息日的規定 (\bibleref{Ex 35:1-3})
    \item \textbf{C`} 會幕的技工和材料 (\bibleref{Ex 35:4-36:7})
    \item \textbf{B`} 建造會幕和祭司的服裝 (\bibleref{Ex 36:8-39:43})
    \item \textbf{A`} 上主的榮耀充滿了會幕 (\bibleref{Ex 40})
  \end{itemize}
\end{frame}

\begin{frame}
  \frametitle{在曠野 \texthebrew{בְּמִדְבַּר}}
  \begin{itemize}
    \item 猶太人把摩西五經分成 54 段,每週讀一段可以在一年內讀完一遍,這叫 \texthebrew{פָּרָשַׁת הַשָּׁבוּעַ}\ \parencite{WeeklyTorahPortion}
    \item 每個分段會用分段第一句話的關鍵字當作名稱,每卷書的第一個分段,同時也是整本書的書名
    \item
      \begin{hebrew}
        וַיְדַבֵּר יְהוָה אֶל־מֹשֶׁה \alert{בְּמִדְבַּר} סִינַי בְּאֹהֶל מוֹעֵד בְּאֶחָד לַחֹדֶשׁ הַשֵּׁנִי בַּשָּׁנָה הַשֵּׁנִית לְצֵאתָם מֵאֶרֶץ מִצְרַיִם לֵאמֹר׃
      \end{hebrew}
  \end{itemize}
\end{frame}

\begin{frame}
  \frametitle{曠野 \texthebrew{מִדְבָּר}\ 與話語 \texthebrew{דָּבָר}}
  \begin{itemize}
    \item 曠野 \texthebrew{מִדְבָּר}\ 是話語 \texthebrew{דָבָר}\ 這個字加上 \texthebrew{מ}\ 前綴,表示地點,工具,或抽象概念。\par
  \end{itemize}
  \centering
  \begin{tabular}{cl}
    \toprule
    原意 & 加上 \texthebrew{מ}\ 前綴 \\
    \midrule
    \texthebrew{מִדְבָּר}\ (曠野) & \texthebrew{דָּבָר}\ (話語) \\
    \texthebrew{גָּדוֹל}\ (大) & \texthebrew{מִגְדָּל}\ (塔) \\
    \texthebrew{שָׁפַט}\ (審判) & \texthebrew{מִשְׁפָּט}\ (法院) \\
    \texthebrew{זָבַח}\ (獻祭) & \texthebrew{מִזְבֵּחַ}\ (祭壇) \\
    \texthebrew{פָּתַח}\ (打開) & \texthebrew{מַפְתֵּחַ}\ (鑰匙) \\
    \texthebrew{מָלַךְ}\ (統治) & \texthebrew{מַמְלָכָה}\ (王國) \\
    \bottomrule
  \end{tabular}
\end{frame}

\begin{frame}{觀察與分享}
  \begin{itemize}
    \item 誰在曠野聽到神的聲音?
    \item 為何在曠野才能聽到神的聲音?
  \end{itemize}
\end{frame}

\begin{frame}
  \frametitle{數點人數 \texthebrew{שְׂאוּ אֶת־רֹאשׁ}}
  \begin{itemize}
    \item \texthebrew{שְׂאוּ אֶת־רֹאשׁ}\ 字面的意思是抬起頭,在聖經中除了統計人數,還有以下的用法:
      \begin{itemize}
        \item 升官: 三天之內,法老必抬起你的頭,恢復你的官位 (\texthebrew{יִשָּׂא פַרְעֹה אֶת־רֹאשֶׁךָ וַהֲשִׁיבְךָ עַל־כַּנֶּךָ})。你仍要把法老的杯送到他的手裏,就像你之前做他司酒長時所做的一樣。(\bibleref{Gn 40:13})
        \item 砍頭: 三天之內,法老必抬起你的頭,把你掛在木頭上 (\texthebrew{יִשָּׂא פַרְעֹה אֶת־רֹאשְׁךָ מֵעָלֶיךָ וְתָלָה אוֹתְךָ עַל־עֵץ});飛鳥要啄食你身上的肉。」(\bibleref{Gn 40:19})
      \end{itemize}
    \item 被神數點也表示被神注意,就像酒政和膳長一樣,最後不一定是好是壞。
    \item \bibleref{2 S 24:1},\bibleref{1 Ch 21:1} 的數點是用不同的字 (\texthebrew{מְנֵה},\texthebrew{לִמְנוֹת})。
  \end{itemize}
\end{frame}

\begin{frame}{上主的榮耀,出埃及記的 \texthebrew{נָשָׂא}\parencite{nasa}}
  「上主,上主!有憐憫、有恩惠的神!不輕易發怒,且有豐盛的慈愛與信實;向千代人信守慈愛,赦免罪孽、過犯與罪惡 (\texthebrew{נֹשֵׂא עָוֺן וָפֶשַׁע וְחַטָּאָה});絕不以有罪的為無罪,必追討父親的罪孽,在兒子身上,及兒子的兒子身上,直到第三代,直到第四代。」(\bibleref{Ex 34:6-7})

  \begin{itemize}
    \item 這裡的赦免 (\texthebrew{נֹשֵׂא}),跟抬起頭 (\texthebrew{שְׂאוּ אֶת־רֹאשׁ})\ 的抬起是同一個字的不同形式,所以赦免可翻譯成背負。
    \item \texthebrew{נֹשֵׂא עָוֺן וָפֶשַׁע וְחַטָּאָה}\ 這四個字同時出現在\bibleref{Is 53}。
  \end{itemize}
\end{frame}

\begin{frame}{觀察與分享}
  \begin{itemize}
    \item 比較\bibleref{Nb 1:7}猶大支派的領袖,以及\bibleref{Nb 34:19}猶大支派的領袖,這兩位是直系血親嗎?
    \item 參考\bibleref{Mt 1:1-17}:
      \begin{itemize}
        \item 大衛是出自哪一系,從民數記到大衛,中間發生了什麼事?
        \item 哪些人在家譜中被跳過了?
        \item 哪些人受到咒詛?(\bibleref{Jr 22:24-30;36:30-31})
        \item 哪位又被揀選?(\bibleref{Zc 1:23})
      \end{itemize}
    \item 從\bibleref{Mt 1:1-17}的家譜中被跳過,咒詛,以及被揀選的人中,思考及分享被神數點(背負)的意義?
  \end{itemize}
\end{frame}

\begin{frame}[allowframebreaks]
  \frametitle{參考文獻}
  \printbibliography
\end{frame}

\end{document}
