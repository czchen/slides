\documentclass{beamer}
\setbeamerfont{normal text}{size=\Huge}

%%%%%%%%%%%%%%%%%%%%%%%%%%%%%%%%%%%%%%%%%%%%%%%%%%%%%%%%%%%%%%%%%%%%%%%%%%%%%%%%
% Fonts
%%%%%%%%%%%%%%%%%%%%%%%%%%%%%%%%%%%%%%%%%%%%%%%%%%%%%%%%%%%%%%%%%%%%%%%%%%%%%%%%

\usepackage{polyglossia}
\setmainlanguage{chinese}
\setotherlanguage{hebrew}
\setotherlanguage{greek}
\setotherlanguage{english}

\usepackage{fontspec}
\usefonttheme{professionalfonts}

\newfontfamily\chinesefont{Noto Sans CJK TC}
\newfontfamily\cjkfontsf{Noto Sans CJK TC}
\newfontfamily\cjkfonttt{Noto Sans CJK TC}

\newfontfamily\hebrewfont[Script=Hebrew]{David CLM}
\newfontfamily\hebrewfontsf[Script=Hebrew]{David CLM}

\newfontfamily\englishfont{Noto Sans}

%%%%%%%%%%%%%%%%%%%%%%%%%%%%%%%%%%%%%%%%%%%%%%%%%%%%%%%%%%%%%%%%%%%%%%%%%%%%%%%%
% Symbol
%%%%%%%%%%%%%%%%%%%%%%%%%%%%%%%%%%%%%%%%%%%%%%%%%%%%%%%%%%%%%%%%%%%%%%%%%%%%%%%%

\usepackage{textcomp}

%%%%%%%%%%%%%%%%%%%%%%%%%%%%%%%%%%%%%%%%%%%%%%%%%%%%%%%%%%%%%%%%%%%%%%%%%%%%%%%%
% Bible References
%%%%%%%%%%%%%%%%%%%%%%%%%%%%%%%%%%%%%%%%%%%%%%%%%%%%%%%%%%%%%%%%%%%%%%%%%%%%%%%%

\usepackage{bibleref-mouth}

\NewDocumentCommand{\setupchinesebible}{m}{
  % 五經
  \providebiblebook{#1}{Genesis}{創世記}
  \providebiblebook{#1}{Exodus}{出埃及記}
  \providebiblebook{#1}{Leviticus}{利未記}
  \providebiblebook{#1}{Numbers}{民數記}
  \providebiblebook{#1}{Deuteronomy}{申命記}

  % 前先知書
  \providebiblebook{#1}{Joshua}{約書亞記}
  \providebiblebook{#1}{Judges}{士師記}
  \providebiblebook{#1}{ISamuel}{撒母耳記上}
  \providebiblebook{#1}{IISamuel}{撒母耳記下}
  \providebiblebook{#1}{IKings}{列王紀上}
  \providebiblebook{#1}{IIKings}{列王紀下}

  % 後先知書 - 大先知書
  \providebiblebook{#1}{Isaiah}{以賽亞書}
  \providebiblebook{#1}{Jeremiah}{耶利米書}
  \providebiblebook{#1}{Ezekiel}{以西結書}

  % 後先知書 - 十二先知書
  \providebiblebook{#1}{Hosea}{何西阿書}
  \providebiblebook{#1}{Joel}{約珥書}
  \providebiblebook{#1}{Amos}{阿摩司書}
  \providebiblebook{#1}{Obadiah}{俄巴底亞書}
  \providebiblebook{#1}{Jonah}{約拿書}
  \providebiblebook{#1}{Micah}{彌迦書}
  \providebiblebook{#1}{Nahum}{那鴻書}
  \providebiblebook{#1}{Habakkuk}{哈巴谷書}
  \providebiblebook{#1}{Zephaniah}{西番雅書}
  \providebiblebook{#1}{Haggai}{哈該書}
  \providebiblebook{#1}{Zechariah}{撒迦利亞書}
  \providebiblebook{#1}{Malachi}{瑪拉基書}

  % 聖卷 - 詩歌書
  \providebiblebook{#1}{Psalms}{詩篇}
  \providebiblebook{#1}{Proverbs}{箴言}
  \providebiblebook{#1}{Job}{約伯記}

  % 聖卷 - 節期書
  \providebiblebook{#1}{SongofSongs}{雅歌}
  \providebiblebook{#1}{Ruth}{路得記}
  \providebiblebook{#1}{Lamentations}{耶利米哀歌}
  \providebiblebook{#1}{Ecclesiastes}{傳道書}
  \providebiblebook{#1}{Esther}{以斯帖記}

  % 聖卷
  \providebiblebook{#1}{Daniel}{但以理書}
  \providebiblebook{#1}{Ezra}{以斯拉記}
  \providebiblebook{#1}{Nehemiah}{尼希米記}
  \providebiblebook{#1}{IChronicles}{歷代志上}
  \providebiblebook{#1}{IIChronicles}{歷代志下}

  % 次經
  \providebiblebook{#1}{Tobit}{多俾亞傳}
  \providebiblebook{#1}{Judith}{友弟德傳}
  \providebiblebook{#1}{IMaccabees}{瑪加伯上}
  \providebiblebook{#1}{IIMaccabees}{瑪加伯下}
  \providebiblebook{#1}{Wisdom}{智慧篇}
  \providebiblebook{#1}{Ecclesiasticus}{德訓篇}
  \providebiblebook{#1}{Baruch}{巴錄書}

  % 福音書
  \providebiblebook{#1}{Matthew}{馬太福音}
  \providebiblebook{#1}{Mark}{馬可福音}
  \providebiblebook{#1}{Luke}{路加福音}
  \providebiblebook{#1}{John}{約翰福音}
  \providebiblebook{#1}{Acts}{使徒行傳}

  % 保羅書信
  \providebiblebook{#1}{Romans}{羅馬書}
  \providebiblebook{#1}{ICorinthians}{哥林多前書}
  \providebiblebook{#1}{IICorinthians}{哥林多後書}
  \providebiblebook{#1}{Galatians}{加拉太書}
  \providebiblebook{#1}{Ephesians}{以弗所書}
  \providebiblebook{#1}{Philippians}{腓立比書}
  \providebiblebook{#1}{Colossians}{歌羅西書}
  \providebiblebook{#1}{IThessalonians}{帖撒羅尼迦前書}
  \providebiblebook{#1}{IIThessalonians}{帖撒羅尼迦後書}
  \providebiblebook{#1}{ITimothy}{提摩太前書}
  \providebiblebook{#1}{IITimothy}{提摩太後書}
  \providebiblebook{#1}{Titus}{提多書}
  \providebiblebook{#1}{Philemon}{腓利門書}
  \providebiblebook{#1}{Hebrews}{希伯來書}

  % 其他書信
  \providebiblebook{#1}{James}{雅各書}
  \providebiblebook{#1}{IPeter}{彼得前書}
  \providebiblebook{#1}{IIPeter}{彼得後書}
  \providebiblebook{#1}{IJohn}{約翰一書}
  \providebiblebook{#1}{IIJohn}{約翰二書}
  \providebiblebook{#1}{IIIJohn}{約翰三書}
  \providebiblebook{#1}{Jude}{猶大書}

  % 啟示錄
  \providebiblebook{#1}{Revelation}{啟示錄}
}

% Fix \textendash
\makeatletter
\providebiblestyle{chinese-text}{\standardbiblestyle{chinese-text}
  {\ }{:}{; }{;}{,}{\textendash}
  {\brm@number@arabic}{\brm@number@arabic}
{#1}{#2}{#3}}
\makeatother

\setupchinesebible{chinese-text}

\providebiblegatewayurl{biblegateway-CNVT}{CNVT}
\providebiblegatewaystyle{chinese}{biblegateway-CNVT}{chinese-text}
\setupchinesebible{chinese}

\setbiblestyle{chinese}

%%%%%%%%%%%%%%%%%%%%%%%%%%%%%%%%%%%%%%%%%%%%%%%%%%%%%%%%%%%%%%%%%%%%%%%%%%%%%%%%
% References
%%%%%%%%%%%%%%%%%%%%%%%%%%%%%%%%%%%%%%%%%%%%%%%%%%%%%%%%%%%%%%%%%%%%%%%%%%%%%%%%

\usepackage[backend=biber,style=numeric,sorting=none,sortcites=true]{biblatex}
\addbibresource{自我介紹.bib}

\begin{document}

\begin{frame}{個人資訊}
  \begin{itemize}
    \item 陳昌倬
    \item 工作: 寫程式
    \item 興趣: 學習技能
      \begin{itemize}
        \item 聖經希伯來文 \parencite{israelbiblicalstudies}
        \item 二等業餘無線電 (BX2ANQ)
        \item 初等救護技術員
        \item 初級防災士
      \end{itemize}
  \end{itemize}
\end{frame}

\begin{frame}{希伯來文}
  \begin{itemize}
    \item 從右到左
    \item 有方塊字 (\texthebrew{כְּתָב אַשּׁוּרִי})\parencite{HebrewAlphabet},手寫字 (\texthebrew{כְּתָב יָד})\parencite{CursiveHebrew},古希伯來文 (\texthebrew{כְּתָב עִבְרִי})\parencite{PaleoHebrew}
    \item 22 個字母,全部都是子音
    \item 聖經中的母音符號 (\texthebrew{נִקּוּד})\parencite{Niqqud} 是由馬索拉學士在六到十世紀發明\parencite{TiberianHebrew},用來標示發音和聲調 (斷句分詞,重音,吟唱)
  \end{itemize}
\end{frame}

\begin{frame}{書讀異文 (\texthebrew{כְּתִיב קְרֵי})}
  \begin{itemize}
    \item \texthebrew{כְּתִיב}\ 是寫下來,\texthebrew{קְרֵי}\ 是念出來
    \item 希伯來聖經中,有些字不是造書寫的文字,而是要用另一個字取代。\parencite{QereAndKetiv}
    \item \texthebrew{יהוה}\ \textrightarrow\ \texthebrew{אֲדֹנָי}\ /\ \texthebrew{אֱלֹהִים}\ /\ \texthebrew{הַשֵּׁם}\par
      \texthebrew{רוּחַ אֲדֹנָי יְהוִה עָלָי יַעַן מָשַׁח יְהוָה אֹתִי}\ (\bibleref{Is 61:1})
    \item \texthebrew{יְרוּשָׁלַם}\ \textrightarrow\ \texthebrew{יְרוּשָׁלַיִם}
  \end{itemize}
\end{frame}

\begin{frame}{字根 (\texthebrew{שֹׁרֶשׁ})}
  字根絕大部分是由三個字母組成,相同字根會衍生出意義相關的字。舉例來說,以下是字根為\ \texthebrew{אמ״ן}\ 的字
  \begin{itemize}
    \item \texthebrew{אֱמֶת}\ 真理\par
      \texthebrew{רֹאשׁ־דְּבָרְךָ אֱמֶת}\ (\bibleref{Ps 119:160})
    \item \texthebrew{אֱמוּנָה}\ 信心\par
      \texthebrew{וְצַדִּיק בֶּאֱמוּנָתוֹ יִחְיֶה׃}\ (\bibleref{Hab 2:4})
    \item \texthebrew{אָמַן}\ 相信\par
      \texthebrew{וְהֶאֱמִן בַּיהוָה וַיַּחְשְׁבֶהָ לּוֹ צְדָקָה׃}\ (\bibleref{Gn 15:6})
    \item \texthebrew{אָמֵן}\par
      \texthebrew{וּבָרוּךְ שֵׁם כְּבוֹדוֹ לְעוֹלָם וְיִמָּלֵא כְבוֹדוֹ אֶת־כֹּל הָאָרֶץ אָמֵן וְאָמֵן׃}\ (\bibleref{Ps 72:19})
    \item \texthebrew{אִמּוּן}\ 練習 (現代希伯來文)
  \end{itemize}
\end{frame}

\begin{frame}{閱讀之母 (Mater lectionis)}
  \begin{itemize}
    \item Mater lectionis (閱讀之母/拉丁文),在希伯來文中,有些子音會當作母音來使用,這種寫法是全寫 (\texthebrew{כְּתִיב מָלֵא}) ,沒有用子音當作母音的是缺陷寫 (\texthebrew{כְּתִיב חֲסֵר})
    \item \texthebrew{דָּוִיד}\ / \texthebrew{דָּוִד}
    \item \texthebrew{קָדוֹשׁ}\ / \texthebrew{קָדֹשׁ}\par
      \texthebrew{וִהְיִיתֶם קְדֹשִׁים כִּי קָדוֹשׁ אָנִי}\ (\bibleref{Lv 11:44})
  \end{itemize}
\end{frame}

\begin{frame}{名詞 (\texthebrew{שֵׁם עֶצֶם})}
  \begin{itemize}
    \item 陰陽性
    \item 單複數
    \item 確定性
    \item 屬格結構 (\texthebrew{סְמִיכוּת})
    \item 所有格後綴
  \end{itemize}
\end{frame}

\begin{frame}{動詞 (\texthebrew{פֹּועַל})}
  \begin{itemize}
    \item 字幹 (\texthebrew{בִּנְיָן})
    \item 構成變化 (\texthebrew{צוּרָה})
    \item 陰陽性
    \item 單複數
    \item 受詞代詞後綴
  \end{itemize}
\end{frame}

\begin{frame}{經文特點}
  \begin{itemize}
    \item \texthebrew{וָאֹמַר הִנְנִי שְׁלָחֵנִי}\ (\bibleref{Is 6:8})
    \item \texthebrew{יְהוָה רֹעִי לֹא אֶחְסָר׃}\ (\bibleref{Ps 23:1})
    \item \texthebrew{אֵלִי אֵלִי לָמָה עֲזַבְתָּנִי}\ (\bibleref{Ps 22:1})
    \item \texthebrew{וְאָהַבְתָּ אֵת יְהוָה אֱלֹהֶיךָ בְּכָל־לְבָבְךָ וּבְכָל־נַפְשְׁךָ וּבְכָל־מְאֹדֶךָ׃}\ (\bibleref{Dt 6:5})
    \item \texthebrew{הַלְלוּ יָהּ הַלְלוּ־אֵל בְּקָדְשׁוֹ הַלְלוּהוּ בִּרְקִיעַ עֻזּוֹ׃}\ (\bibleref{Ps 150:1})
  \end{itemize}
\end{frame}

\begin{frame}{以希伯來文分析經文}
  以下是摩西在約旦河東的曠野,疏弗對面 (\texthebrew{מוֹל סוּף}) 的亞拉巴 (\texthebrew{בָּעֲרָבָה}),就是在巴蘭和陀弗、拉班 (\texthebrew{וּבֵין־תֹּפֶל וְלָבָן})、哈洗錄、底撒哈 (\texthebrew{וְדִי זָהָב}) 之間,向以色列人所說的話。(\bibleref{Dt 1:1})\parencite{Devarim}
\end{frame}

\begin{frame}[allowframebreaks]
  \frametitle{參考文獻}
  {\footnotesize\printbibliography}
\end{frame}

\end{document}
