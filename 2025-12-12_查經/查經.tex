\documentclass{beamer}

\usetheme{CambridgeUS}
\usecolortheme{orchid}

%%%%%%%%%%%%%%%%%%%%%%%%%%%%%%%%%%%%%%%%%%%%%%%%%%%%%%%%%%%%%%%%%%%%%%%%%%%%%%%%
% Fonts
%%%%%%%%%%%%%%%%%%%%%%%%%%%%%%%%%%%%%%%%%%%%%%%%%%%%%%%%%%%%%%%%%%%%%%%%%%%%%%%%

\usepackage{polyglossia}
\setmainlanguage{chinese}

\usepackage{fontspec}
\usefonttheme{professionalfonts}

\newfontfamily\chinesefont{Noto Sans CJK TC}
\newfontfamily\cjkfontsf{Noto Sans CJK TC}
\newfontfamily\cjkfonttt{Noto Sans CJK TC}

%%%%%%%%%%%%%%%%%%%%%%%%%%%%%%%%%%%%%%%%%%%%%%%%%%%%%%%%%%%%%%%%%%%%%%%%%%%%%%%%
% Bible
%%%%%%%%%%%%%%%%%%%%%%%%%%%%%%%%%%%%%%%%%%%%%%%%%%%%%%%%%%%%%%%%%%%%%%%%%%%%%%%%

\usepackage{bibleref-mouth}
\providebiblebookalias{Psalms}{Psalm} % Support Psalm as alias

\newcommand*{\setupchinesebible}[1]{
  % 五經
  \providebiblebook{#1}{Genesis}{創世記}
  \providebiblebook{#1}{Exodus}{出埃及記}
  \providebiblebook{#1}{Leviticus}{利未記}
  \providebiblebook{#1}{Numbers}{民數記}
  \providebiblebook{#1}{Deuteronomy}{申命記}

  % 前先知書
  \providebiblebook{#1}{Joshua}{約書亞記}
  \providebiblebook{#1}{Judges}{士師記}
  \providebiblebook{#1}{ISamuel}{撒母耳記上}
  \providebiblebook{#1}{IISamuel}{撒母耳記下}
  \providebiblebook{#1}{IKings}{列王紀上}
  \providebiblebook{#1}{IIKings}{列王紀下}

  % 後先知書 - 大先知書
  \providebiblebook{#1}{Isaiah}{以賽亞書}
  \providebiblebook{#1}{Jeremiah}{耶利米書}
  \providebiblebook{#1}{Ezekiel}{以西結書}

  % 後先知書 - 十二先知書
  \providebiblebook{#1}{Hosea}{何西阿書}
  \providebiblebook{#1}{Joel}{約珥書}
  \providebiblebook{#1}{Amos}{阿摩司書}
  \providebiblebook{#1}{Obadiah}{俄巴底亞書}
  \providebiblebook{#1}{Jonah}{約拿書}
  \providebiblebook{#1}{Micah}{彌迦書}
  \providebiblebook{#1}{Nahum}{那鴻書}
  \providebiblebook{#1}{Habakkuk}{哈巴谷書}
  \providebiblebook{#1}{Zephaniah}{西番雅書}
  \providebiblebook{#1}{Haggai}{哈該書}
  \providebiblebook{#1}{Zechariah}{撒迦利亞書}
  \providebiblebook{#1}{Malachi}{瑪拉基書}

  % 聖卷 - 詩歌書
  \providebiblebook{#1}{Psalms}{詩篇}
  \providebiblebook{#1}{Proverbs}{箴言}
  \providebiblebook{#1}{Job}{約伯記}

  % 聖卷 - 節期書
  \providebiblebook{#1}{SongofSongs}{雅歌}
  \providebiblebook{#1}{Ruth}{路得記}
  \providebiblebook{#1}{Lamentations}{耶利米哀歌}
  \providebiblebook{#1}{Ecclesiastes}{傳道書}
  \providebiblebook{#1}{Esther}{以斯帖記}

  % 聖卷
  \providebiblebook{#1}{Daniel}{但以理書}
  \providebiblebook{#1}{Ezra}{以斯拉記}
  \providebiblebook{#1}{Nehemiah}{尼希米記}
  \providebiblebook{#1}{IChronicles}{歷代志上}
  \providebiblebook{#1}{IIChronicles}{歷代志下}

  % 次經
  \providebiblebook{#1}{Tobit}{多俾亞傳}
  \providebiblebook{#1}{Judith}{友弟德傳}
  \providebiblebook{#1}{IMaccabees}{瑪加伯上}
  \providebiblebook{#1}{IIMaccabees}{瑪加伯下}
  \providebiblebook{#1}{Wisdom}{智慧篇}
  \providebiblebook{#1}{Ecclesiasticus}{德訓篇}
  \providebiblebook{#1}{Baruch}{巴錄書}

  % 福音書
  \providebiblebook{#1}{Matthew}{馬太福音}
  \providebiblebook{#1}{Mark}{馬可福音}
  \providebiblebook{#1}{Luke}{路加福音}
  \providebiblebook{#1}{John}{約翰福音}
  \providebiblebook{#1}{Acts}{使徒行傳}

  % 保羅書信
  \providebiblebook{#1}{Romans}{羅馬書}
  \providebiblebook{#1}{ICorinthians}{哥林多前書}
  \providebiblebook{#1}{IICorinthians}{哥林多後書}
  \providebiblebook{#1}{Galatians}{加拉太書}
  \providebiblebook{#1}{Ephesians}{以弗所書}
  \providebiblebook{#1}{Philippians}{腓立比書}
  \providebiblebook{#1}{Colossians}{歌羅西書}
  \providebiblebook{#1}{IThessalonians}{帖撒羅尼迦前書}
  \providebiblebook{#1}{IIThessalonians}{帖撒羅尼迦後書}
  \providebiblebook{#1}{ITimothy}{提摩太前書}
  \providebiblebook{#1}{IITimothy}{提摩太後書}
  \providebiblebook{#1}{Titus}{提多書}
  \providebiblebook{#1}{Philemon}{腓利門書}
  \providebiblebook{#1}{Hebrews}{希伯來書}

  % 其他書信
  \providebiblebook{#1}{James}{雅各書}
  \providebiblebook{#1}{IPeter}{彼得前書}
  \providebiblebook{#1}{IIPeter}{彼得後書}
  \providebiblebook{#1}{IJohn}{約翰一書}
  \providebiblebook{#1}{IIJohn}{約翰二書}
  \providebiblebook{#1}{IIIJohn}{約翰三書}
  \providebiblebook{#1}{Jude}{猶大書}

  % 啟示錄
  \providebiblebook{#1}{Revelation}{啟示錄}
}

% Fix \textendash
\makeatletter
\providebiblestyle{chinese-text}{\standardbiblestyle{chinese-text}
  {\ }{:}{; }{;}{,}{\textendash}
  {\brm@number@arabic}{\brm@number@arabic}
{#1}{#2}{#3}}
\makeatother

\setupchinesebible{chinese-text}

\providebiblegatewayurl{biblegateway-CUV}{CUV}
\providebiblegatewaystyle{chinese}{biblegateway-CUV}{chinese-text}
\setupchinesebible{chinese}

\setbiblestyle{chinese}

%%%%%%%%%%%%%%%%%%%%%%%%%%%%%%%%%%%%%%%%%%%%%%%%%%%%%%%%%%%%%%%%%%%%%%%%%%%%%%%%
% Misc
%%%%%%%%%%%%%%%%%%%%%%%%%%%%%%%%%%%%%%%%%%%%%%%%%%%%%%%%%%%%%%%%%%%%%%%%%%%%%%%%

\title{信友團契 2025 查經 Lesson 12}
\subtitle{\bibleref{2Th 3:1-18}:按規矩而行}
\date{2025-12-12}

\begin{document}

\begin{frame}
  \titlepage
\end{frame}

\begin{frame}
  \frametitle{\bibleref{2Th 3:1-9}}
  \textbf{1}\ 弟兄們,我還有話說,請你們為我們禱告,好叫主的道理快快行開,得著榮耀,正如在你們中間一樣。\textbf{2}\ 也叫我們脫離無理之惡人的手,因為人不都是有信心。\textbf{3}\ 但主是信實的,要堅固你們,保護你們脫離那惡者〔或作:脫離兇惡〕。\textbf{4}\ 我們靠主深信,你們現在是遵行我們所吩咐的,後來也必要遵行。\textbf{5}\ 願主引導你們的心,叫你們愛神,並學基督的忍耐。\textbf{6}\ 弟兄們,我們奉主耶穌基督的名吩咐你們,凡有弟兄不按規矩而行,不遵守從我們所受的教訓,就當遠離他。\textbf{7}\ 你們自己原知道應當怎樣效法我們,因為我們在你們中間,未嘗不按規矩而行。\textbf{8}\ 也未嘗白吃人的飯,倒是辛苦勞碌,晝夜作工,免得叫你們一人受累。\textbf{9}\ 這並不是因我們沒有權柄,乃是要給你們作榜樣,叫你們效法我們。
\end{frame}

\begin{frame}
  \frametitle{\bibleref{2Th 3:10-18}}
  \textbf{10}\ 我們在你們那裡的時候,曾吩咐你們說,若有人不肯作工,就不可吃飯。\textbf{11}\ 因我們聽說,在你們中間有人不按規矩而行,甚麼工都不作,反倒專管閒事。\textbf{12}\ 我們靠主耶穌基督,吩咐勸戒這樣的人,要安靜作工,吃自己的飯。\textbf{13}\ 弟兄們,你們行善不可喪志。\textbf{14}\ 若有人不聽從我們這信上的話,要記下他,不和他交往,叫他自覺羞愧。\textbf{15}\ 但不要以他為仇人,要勸他如弟兄。\textbf{16}\ 願賜平安的主,隨時隨事親自給你們平安!願主常與你們眾人同在!\textbf{17}\ 我保羅親筆問你們安·凡我的信都以此為記,我的筆跡就是這樣。\textbf{18}\ 願我們主耶穌基督的恩,常與你們眾人同在!
\end{frame}

\begin{frame}
  \frametitle{討論問題 Q1,Q2,Q3}
  \textbf{1}\ 弟兄們,我還有話說,請你們為我們禱告,好叫主的道理快快行開,得著榮耀,正如在你們中間一樣。\textbf{2}\ 也叫我們脫離無理之惡人的手,因為人不都是有信心。
  \begin{itemize}
    \item \textbf{Q1}\ 根據帖後 3:1-2,經文說到保羅指的「惡人」是誰?
    \item \textbf{Q2}\ 根據帖後 3:1-2,保羅在第 1 節請求為他們禱告,求「主的道快快行開,得著榮耀」。對你來說,在台北這個城市,你認為「主的道」在哪些方面最需要「行開」和「得榮耀」?
    \item \textbf{Q3}\ 根據帖後 3:1-2,保羅期望帖撒羅尼迦人在哪兩方面的事情上為他禱告?
  \end{itemize}
\end{frame}

\begin{frame}
  \frametitle{討論問題 Q4,Q5}
  \textbf{3}\ 但主是信實的,要堅固你們,保護你們脫離那惡者〔或作:脫離兇惡〕。\textbf{4}\ 我們靠主深信,你們現在是遵行我們所吩咐的,後來也必要遵行。\textbf{5}\ 願主引導你們的心,叫你們愛神,並學基督的忍耐。\textbf{6}\ 弟兄們,我們奉主耶穌基督的名吩咐你們,凡有弟兄不按規矩而行,不遵守從我們所受的教訓,就當遠離他。
  \begin{itemize}
    \item \textbf{Q4}\ 根據帖後 3:3-5,當保羅請求帖撒羅尼迦人為他禱告時,他如何表達他在主裡的信心?
    \item \textbf{Q5}\ 據帖後 3:6,保羅用很重的語氣說:「奉主耶穌基督的名吩咐你們,遠離每一位不按規矩而行,也不遵守所受傳統的弟兄」。在職場或團契裡,我們有沒有看過「不按規矩而行」的人?這會帶來什麼影響?
  \end{itemize}
\end{frame}

\begin{frame}
  \frametitle{討論問題 Q6,Q7,Q8,Q12}
  \textbf{7}\ 你們自己原知道應當怎樣效法我們,因為我們在你們中間,未嘗不按規矩而行。\textbf{8}\ 也未嘗白吃人的飯,倒是辛苦勞碌,晝夜作工,免得叫你們一人受累。\textbf{9}\ 這並不是因我們沒有權柄,乃是要給你們作榜樣,叫你們效法我們。

  \begin{itemize}
    \item \textbf{Q6}\ 保羅說「你們卻要效法我們」(3:7,9)。在團契裡,我們有沒有活出讓別人可以 「效法」的見證?尤其在「工作態度,金錢觀,時間管理上?
    \item \textbf{Q7}\ 據帖後 3:8-9,保羅說自己「在你們那裡的時候,並沒有白吃人的飯……總要勞力,親手做工」。保羅為什麼特別強調「不白吃飯」?這對我們今天的職場基督徒有什麼提醒?
    \item \textbf{Q8}\ 據帖後 3:7-9,保羅如何以身作則,說明對待工作的正確態度?
    \item \textbf{Q12}\ 據帖後 3:8,保羅提到他們親自「晝夜做工,辛勤勞碌」來樹立榜樣。在你的工作領域或服事中,你願意讓別人看到你的哪些工作態度作為「按規矩而行」的榜樣?
  \end{itemize}
\end{frame}

\begin{frame}
  \frametitle{討論問題 Q9}
  \textbf{6}\ 弟兄們,我們奉主耶穌基督的名吩咐你們,凡有弟兄不按規矩而行,不遵守從我們所受的教訓,就當遠離他。\textellipsis{}\textbf{14}\ 若有人不聽從我們這信上的話,要記下他,不和他交往,叫他自覺羞愧。\textbf{15}\ 但不要以他為仇人,要勸他如弟兄。
  \begin{itemize}
    \item \textbf{Q9}\ 保羅對這群「不做工」的人,既說要「遠離」(3:6,14),又說「不可當仇敵,乃要勸他如兄弟」(3:15)。你覺得這兩件事要怎麼同時做到?有沒有實際可以操作的做法?
  \end{itemize}
\end{frame}

\begin{frame}
  \frametitle{討論問題 Q10,Q13}
  \textbf{12}\ 我們靠主耶穌基督,吩咐勸戒這樣的人,要安靜作工,吃自己的飯。\textbf{13}\ 弟兄們,你們行善不可喪志。
  \begin{itemize}
    \item \textbf{Q10}\ 據帖後 3:12,保羅對不工作的人發出一個命令:「要安靜做工,吃自己的飯」。你如何平衡「努力工作」與「安靜」這兩個看似矛盾的要求?在你的職場/生活,如何實踐「安靜做工」?
    \item \textbf{Q13}\ 據帖後 3:13,保羅勸勉信徒「你們行善不可喪志」。在教會團契生活中,你做過哪些「善事」(例如:服事、關心人)卻讓你感到喪志(疲憊、不被感謝、或沒有看到果效)?回想這些事情,你有什麼學習的心得?
  \end{itemize}
\end{frame}

\begin{frame}
  \frametitle{討論問題 Q11,Q14}
  \begin{itemize}
    \item \textbf{Q11}\ 讀完這一章之後,未來三個月,你想在「忠心工作+愛心管教」這件事上,有哪一個具體的行動改變計畫?
    \item \textbf{Q14}\ 請分享今天的查經經文中你最有感動和領受的經文。
  \end{itemize}
\end{frame}

\end{document}
